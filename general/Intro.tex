The reconstruction and calibration can be loosely split in two phases:

\begin{itemize}

\item Online: performed during the data taking on the data received from the
upstream components of the processing chain or short time buffers.
While the primary aim of these operations is to achieve a data
compression level acceptable for the permanent storage at \TBD{$\sim$12.4
GB/s in average and $\sim$75 GB/s peak rate}~\cite{panelDoc}, they
should also
facilitate as much as possible further processing of data to analysis
grade level. 
The calibrations performed at this phase should minimize
the loss of the useful data during the lossy compression steps and
preferably allow later reconstruction for physics analysis without
special calibration passes.

\item Offline: decoupled from the data taking process, using as an input the
data from the permanent storage. Its aim is to provide analysis grade
level for the full detector, analyzable in terms of separate
collisions.
\end{itemize}

The online reconstruction and calibration proceed in two phases on:

\begin{itemize}
\item First Level Processors (FLP), receiving raw data from the
Detector Data Links (DDL). Each FLP sees all data (e.g. all
time-slices, without omissions)
from only the part of detector it services. The data from FLP's are 
aggregated and split into pieces containing information from all detectors for
predefined time-frames (\TF).
\item Event Processing Nodes (EPN), which process \TF's
  asynchronously.
\end{itemize}

