\subsection{ZDC calibration and reconstruction}
\label{ZDC:FLP}

For calibration purposes, the pedestal subtraction is no more needed since 
the baseline will be automatically subtracted from the signal. 
Concerning the energy calibration, similarly to what done in previous runs, 
the ZDC needs a significant sample of events using a trigger provided by the ZDC itself. 
The aim is to collect events coming from electromagnetic dissociation processes where the 
dominant emission channel results in the emission of single nucleons, allowing to perform 
the energy calibration and also to estimate aging effects on the detector. 
These runs will be standalone runs that should last few minutes and the needed frequency 
would be not less than every 2/3 days to monitor possible signal variation and resolution 
worsening due to aging.

The reconstruction will be similar to the current one, but simpler and faster since the code 
involving the pedestal subtraction will be avoided and the reconstruction algorithm will 
essentially consist in the energy calibration \TBD{in these standalone run and using the obtained
calibration coefficients for the energy estimates in the physics runs ???}
