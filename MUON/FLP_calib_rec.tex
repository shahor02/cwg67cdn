\subsection{MUON}
\label{MUON:FLP}

%-----------------------------------------------------------------------------
\subsubsection{CRU}

Depending on the CRU FPGA available space and the availability of
competent (wo)manpower, we may think of reorganizing the MCH data in
ways that could help the pre-clustering stage performed in FLPs, or
even to do a pre-clustering on the CRU. As it's unclear whether all
this is doable at all, we will in this document assume nothing special
is done in the CRU.

\subsubsection{FLP}

On the FLP the very minimal set of operations that are foreseen
includes occupancy (MCH) and noisy channel (MID) maps computation.

Very likely, we should also be able to perform pre-clustering on the
FLPs. The pre-clustering is a simple clustering algorithm which groups
adjacent pads, irrespective of their charges, into preclusters. The
current (recursive) implementation used in AliRoot 5 accounts for
roughly 20\% of the total clustering time. Very initial investigations
about possible optimizations of the way we access/store pad
information (e.g. using more static data structures with respect to
the current ones) have shown that we can most certainly gain a factor
20 in speed for the pre-clustering stage, getting to about 2 ms per MB
\pbpb event on one core of an Intel i7 @ 2.3 GHz processor (excluding
I/O). Further speedup would require moving part of the pre-clustering
into accelerator cards, e.g. FPGAs (hosted by the CRU or the FLPs
themselves, whatever the case may be). Manpower to investigate this
option is being searched for.

Whether or not the next processing stage, namely the clustering, can
fit within the FLPs, is an opened question. We first have to assume
that the MCH FLPs will see complete detection elements (either slats
or quadrants), otherwise we don't have the relevant information for
clustering. Then, if we assume also that they'll be equipped with GPUs
(or other "accelerator" like Xeon Phi) then the plan is to convert the
clustering stage to a GPU version, where we can only guess to be able
to gain a factor 50 in speed. Admittedly, without even an early
prototype of that version, the speedup factor is more a wish than a
real estimate... But even without a GPU version we expect to be able
to improve the current CPU implementation, e.g. using the same static
data structures as in the pre-clustering. Whether or not such an
implementation would fit in a FLP remains to be demonstrated.

For MID, it is still unclear whether or not anything will have to be
developed at the FLP level. In the case we are able to reconstruct at
the EPN level all the MCH tracks (see next section) then nothing is
really needed for MID at FLP level. On the contrary, if we need an a
priori selection of the events to be fully reconstructed by MCH (in
the case the full reconstruction is found to be too slow to cope with
50 kHz event rate), then we will need to get the computation of a
trigger-like decision somewhere in the chain. We know we will get
exactly one FLP for MID, so that FLP will see all the MID data at
once, and hence can compute a trigger-like decision. Given the current
uncertainty of the achievable speed of the full MCH reconstruction, we
do plan to investigate, during Run2 already (within the HLT farm), an
implementation of the trigger decision, that could be used to select
which events should be fully reconstructed.

The data reduction that can be achieved at FLP level is currently
largely unknown. Lossless compression (using e.g. Huffman coding) of
the pad data has not been investigated yet. In the worst case, we will
in fact slightly increase (by $\sim$ 3\%) the data volume (after pad cleaning), as we'll be
adding some data to mark the pre-clusters : we would for instance
re-order the digits by pre-cluster and thus need some boundary words (assuming 16 bits would be enough)
to separate the digits from different pre-clusters. This may tell us that the clustering
should not be done at the FLP level, even if fast enough, but on EPNs.
