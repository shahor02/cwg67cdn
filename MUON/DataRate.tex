\subsection{MUON}
\label{MUON:datarate}

Data rate estimates for MUON are based on past observations from 
Run1 \pbpb runs. Even though we expect an increase in particle 
multiplicities (by $\sim$ 25\%), we know that the Muon Tracker (MCH) 
data size is to a large extent dominated by FEE noise, and 
that Muon Trigger (MTR currently that will become the MID, 
Muon Identifier) data size is a constant.  
As we expect that the new Muon Tracker FEE will solve the 
stability problems, we then assume that Run1 data 
sizes are a very reasonable estimate of Run3 data sizes. 
The Run1 data was heavily biased towards central events 
due to the trigger used, so the average event size was 
biased towards high values (60 kB/event for MCH). A more 
detailed analysis reveals that MB events are around 
25 kB/event and central ones around 75 kB/event. 
MID events are expected to be 3 kB/event (compared to the 
current 7 kB/event). So all in all the expected Muon data 
rate is $\sim$ 1.5 GB/s for 50~kHz MB.

Note that the upgrade of the MCH readout itself is actually 
designed to handle a much higher data rate\cite{Run3OnlineTDR}, 
as it must accept peak rates. Those have been estimated 
to 100 GB/s at 50~kHz, assuming peak occupancies of 9\% and a 
readout mode where all the ADC samples (10 of them per pad) are 
readout, for a total of 140 bits per pad. That mode of operation is to 
be considered a debug mode and will not be used in a sustained way. 
The normal, routine, readout mode will have only one charge and address 
per channel which amount to 90 bits per channel (per the SAMPA specification). 
We assume at this stage that it will be possible to reduce this (in the CRU, for instance) 
to 64 bits per channel.
