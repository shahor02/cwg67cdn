\subsection{ITS}
\label{ITS:datarate}

Table~\ref{tab:ITSrate} summarizes the hit rate in the ITS expected in continuous readout with
30\ums frames at 50kHz \pbpb collisions, including the contribution from the QED electrons and 
random noise at the rate of $10^{-5}$ per pixel in $2.5\times 10^{10}$ channels. 
Minimum bias collision multiplicity is assumed to be 1/3 of that in
the central collision

\begin{table}[ht]
\begin{tabular}{ | p{0.8cm} | p{1.6cm} | p{1.6cm} | p{1.8cm} | p{2cm} | p{3cm} | }
%\begin{tabular}{ | l | l | l | l | l | l | }
\hline
   & \multicolumn{3}{|c|}{Single central \pbpb collision} & QED 30 \ums @ 50kHz & Min.bias \pbpb 30 \ums @ 50kHz \\ \hline
Lr & Min per sensor & Max per sensor & Mean per layer & Mean per layer      & Mean per layer        \\ \hline
0  & 31             & 137            & 9948           & 8400                & 13400                 \\ \hline
1  & 30             & 92             & 9468           & 5400                & 10100                 \\ \hline
2  & 26             & 67             & 8978           & 3400                &  7900                 \\ \hline
3  & 1.7            & 4.7            & 8732           & 120                 &  4500                 \\ \hline
4  & 1.4            & 3.0            & 7717           & 72                  &  3900                 \\ \hline
5  & 0.62           & 1.55           & 9336           & 52                  &  4700                 \\ \hline
6  & 0.55           & 1.22           & 8819           & 44                  &  4500                 \\ \hline
\multicolumn{5}{|c|}{Total hit clusters per 30 \ums frame}                  & 49000                 \\ \hline
\multicolumn{5}{|c|}{Noisy pixels in 2.5$\times 10^{10}$ channels}          & 250000                \\ \hline
\hline
\end{tabular}
\caption{Hit rates from hadronic collisions and QED electrons. Random noise rate assumed to be $10^{-5}$ per pixel per 30\ums 
readout frame.}
\label{tab:ITSrate}
\end{table}

As one can see, the data rate is dominated by the noise, which
produces only single-pixel clusters. According to simulations,
the single-pixel cluster probability from real hits is $\sim 1.2\%$, 
and $\sim 70\%$ of such pixels are in fact split from larger
clusters produced by the same particle. Therefore, one can consider
discarding single-pixel clusters either in the FE or during the
clusterization on the FLPs. But, given that the detector design is not
final yet, the fraction of real single-pixel clusters may 
increase (particularly, for larger pixel pitch). For the conservative estimate of ITS data rate on the 
input to the online system we assume that the single-pixel clusters (if
not suppressed) will be packed using 20 bits per cluster while
remaining hits will consume in average 40 bits per clusters (by
storing the row, column and 4$\times$4 pixels plaquette (partially)
containing the cluster. This leads to $\sim$ 30 (8.2) GB/s at 50~kHz and
$\sim$ 38 (16.4) GB/s at 100~kHz without (with) suppression of the single-pixel
clusters.
