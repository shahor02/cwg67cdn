\subsection{TOF Time Calibration}
\label{sec:TOFTimeCalibration}
The collection of the information needed to perform TOF time calibration 
can start only at the stage when the TPC tracking performance is sufficiently stable and accurate to allow
reliable extrapolation to TOF and matching with TOF signals. The TOF time calibration includes the measurement
of a few time offsets with different granularity, of which single channel offsets and channel time slewing parameters are 
the most demanding in terms of statistics. These parameters are expected to be constant in time, being related only
to the hardware configuration, and can be obtained in the foreseen commissioning phase at the beginning of Run3. 
On the other hand, it cannot be excluded that hardware interventions may happen, requiring, as a consequence, 
an update of such parameters. For this reason, different approaches are being investigated in order to keep 
track of the calibration information which might be needed for update and adjustment. One possibility could be 
to store on a run by run basis histograms or trees, with a preselection on the tracks. 

The same calibration information will in addition be used in order to monitor the behaviour of the detector
by looking at the distribution of the time signals per channel. Channels whose signal is unstable or 
badly behaving should not be used for physics analysis . For instance, this could be due to an unexpected 
loss of configuration of the electronics or to data corruption not tracked down during the previous calibration 
steps~\ref{sec:TOFElectronicsStatus}.
%Still under investigation which algorithm could be implemented so that this is done automatically. 

The possibility to use the time calibration only at analysis level is being evaluated. This would imply that in case TOF
signals from adjacent channels (generated by the same traversing particle) have to be merged in a single
cluster, this can only happen at analysis. 

