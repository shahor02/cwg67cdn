\subsection{TRD calibration and reconstruction}
\label{TRD:FLP}

%-----------------------------------------------------------------------------
\subsubsection{Reconstruction}

The TRD reconstruction will have to follow different schemas,
depending on which type of readout is employed.  

In the case of the full raw ADC data readout or the partial data
readout, the it will follow essentially the same sequence as has been
used up to now.  It starts with cluster finding based on the raw data
for all read time bins, followed by the offline tracklet finding
(layer-by-layer).  This tracklet information is then the basis of the
global track finding using the Kalman filter.  For Run-3 the first two
steps (cluster and tracklet finding) might be run of the FLP, while
the track finding will be part of the EPN stage.

If the readout mode is changed to the tracklet mode, the
reconstruction is foreseen to be based on the online tracklet
information only.  First studies indicate that this is feasible
without severe loss in PID performance.  Since this mode does not
require any offline cluster and tracklet finding, the FLP stage would
only be needed for post-processing the online tracklets (e.g. quality
selection, marking of pad row crossings, etc.).  The data rate between
FLP and EPN would therefore be only slightly lower than the input rate
to FLP in this case.  Since this reconstruction mode needs a reliable
extrapolation from other detectors (TPC), a very good alignment is
mandatory.

Running the TRD in a mixed mode would have the consequence that the
reconstruction should be able to apply both scenarios sketched above
to these events, since they contain full zero-suppressed ADC data as
well as the tracklet information.

In order to improve the TPC calibration procedure, the TRD can be used
to provide a first estimate on possible track candidates.  This will
require the implementation of a TRD track finding in a quasi
stand-alone mode, that will need to be developed in detail still.

%-----------------------------------------------------------------------------
\subsubsection{Calibration}

Currently, the TRD calibration procedure is implemented for the
following quantities: gain factors, drift velocity $v_{d}$,
ExB-effect, time offset $t_{0}$, and chamber status.  These procedures
could be used in the same way on the EPN for the full and partial
readout modes.  Especially, the fraction of events fully read out in
the mixed mode could be calibrated according to the established
procedures and should provide sufficient statistics for an accurate
calibration.

In the tracklet readout mode, a tail cancellation of the input data
and a gain calibration can be performed (possibly on FLPs) based on
the charge information included in the online tracklets.  However, a
calibration of other parameters ($t_{0}$ and $v_{d}$) is not directly
possible, but might be replaced by a global alignment procedure.


