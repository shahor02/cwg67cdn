\subsection{FIT calibration and reconstruction}
\label{FIT:FLP}

FIT calibration consist of 3 procedures:
\begin{itemize}
\item Slewing correction – required typically once per year; it is done either with the aid of the laser calibration 
system or/and with the physics data collected during the 1st day of data taking
\item Channel equalization – it is normally based on the first 20000 events; equalization can be done on FLP once a week
\item Global offsets – required to place T0A and T0C around zero. Calibration of these offsets is time-dependent 
and need information about full event. It can be performed on the FLP provided that the whole detector is processed
by the single FLP, otherwise, should be performed on the EPN.
\end{itemize}

The main purpose of reconstruction of the FIT data is to improve the precision of the
trigger parameters that are generated and available already online:
\begin{itemize}
\item Interaction time: calculated in FE as the average of the arrival time of the 1st signal generated on the 
A and C sides: (T0A+T0C)/2. Parameter equalization for trigger generation must be hardwired and therefore 
it’s accuracy is limited 
by the step size of the delay boxes and stability of the electronic units. 
During the reconstruction the delays can be equalized to higher precision and the interaction 
time can be extracted not just from the first signals but from the average arrival time on each side
(high precision interaction time is needed for particle identification using the TOF method).

\item The vertex position: defined as (T0A-T0C)/2. For trigger generation T0A and T0C are just the first 
signals on each side and instead of the numerical value only a binary value is given indicating the 
location of the interaction point either inside or outside of the predefined region. 
Reconstruction is needed to obtain the numerical value of the vertex position. 

\item The online multiplicity is only rudimentary and is accomplished with two pre-fixed comparator values 
indicating reaching the semi-central and central collision thresholds. 
During the reconstruction the sum of the amplitudes on each MCP sector can be extracted with a good 
precision giving not only the total value but also the pseudo rapidity distribution and the 
indication of the reaction plane. 
\end{itemize}

The reconstruction of the FIT takes negligible time and can be performed on the FLP (provided that both sides are
processed by the same FLP).

