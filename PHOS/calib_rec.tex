\subsection{PHOS calibration and reconstruction}
\label{PHOS:FLP}

%-----------------------------------------------------------------------------
\subsubsection{Reconstruction}

The ultimate goal of PHOS raw data reconstruction needed for further
physics analysis is producing reconstructed showers developed by
particles in the PHOS medium, with each cluster characterized by its
energy, coordinate, time and shower shape parameters. In general, PHOS
raw data reconstruction consists of two steps:
%
\begin{enumerate}
  \item Reconstruction of the amplitude and time of cells from the
    sampled digitized signal. It leads to a significant reduction of
    data volume, as from an array of up to 30 10-bit samples, only two
    floating-point parameters are derived, cell energy $E$ and time
    stamp $t$. In addition, a signal shape quality parameter can be
    derived from the fitting procedure which can be further used to
    assess the signal quality.
  \item Reconstruction of clusters of adjacent cells in PHOS. Each
    cluster, representing the shower developed by a particle
    interacting with the PHOS medium, is characretized by the
    reconstructed cluster energy, time with respect to the L0 trigger
    instance, cluster coordinate calculated from the cluster gravity
    center, and cluster shape parameters used later in analysis for
    photon identification.
\end{enumerate}
%
Further physics analysis operates with clusters which are used to
evaluate the 4-momenta of reconstructed photons and to calculate
identification probabilities to be a photon shower of clusters
produced by other type of particles.

Among possible implementation of on-line reconstruction procedures at
the FLP level, one can consider reconstruction of cell energy and time
from the sampled digitized data. It might be necessary to
significantly reduce the data payload produced by the PHOS detector to
the whole raw data stream. Estimated reduction of event size by this
signal shape reconstruction compared to full sampled raw data is a
factor of 10. This signal shape reconstruction can be potentially
implemented either at the FPGA of the PHOS front-end electronics, or
in HLT. Prototypes of HLT reconstruction are available in
\verb|aliroot| since Run-1.

Reconstruction of cluster from cells is not considered as an option
for physics analysis. Having cells in reconstructed data is necessary
for further offline calibration, also needed for tuning cluster fining
algorithms, for reconstruction efficiency calculation via embedding,
i.e. via merging simulated signals with real raw data. Cluster finding
cannot work in FLP, as cluster fining requires access to cells which
might be detected in different detector partitions.

%-----------------------------------------------------------------------------
\subsubsection{Calibration}

PHOS energy and timing calibration requires very large statistics
which cannot be accumulated in one physics run. Calibration is based
on physics observables such as $\pi^0$ peaks on invariant mass
spectra, average deposited energy in cells or time of cluster of
particles from the interaction point to PHOS. Although calibration
algorithms themselves are not feasible in on-line, filling and
accumulating histograms during data taking and storing them at an
external data storage might help in further offline calibration. 
