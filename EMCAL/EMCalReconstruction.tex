\subsection{EMCal Reconstruction}

The energy deposited by the particles in the towers produces scintillating light that is propagated with optic fibers through the different layers to APD placed at the base of the cells. The APDs amplify the signal and generate an electronic pulse shape that is stored in the raw data format which consists of a series of so-called time samples with 10-bit ADC counts per channel. Each time bin is 100 ns wide, corresponding to a 10 MHz readout.
The signal has a Gamma-2 shape. From this pulse shape, we extract the signal amplitude and the arrival time. The pulse shape is fitted (\bf{fitter to be chosen and described here? right now, Run1-2 TMinuit fitter, for Run3 use  HLT fast fitter  AliCaloRawAnalyzerPeakFinder?}), and those 2 values are extracted and form what is called as digits.

A particle produces signals in different towers (electromagnetic shower expands more than its Moli\`ere radius which is a cell size). The next step is the formation of clusters of cells that belong to the same particle, although depending on the energy, granularity, clusterization algorithm or event type, those clusters might have contributions from different particles. The default algorithm in pp collisions is a simple aggregation of neighboring cells until there is no more cells above a certain energy threshold (named {\it clusterizer V1}). In case of Pb-Pb collisions environment, where particle showers merge quite often, we apply another algorithm that aggregates cells to the clusters until reaching a cell with more energy than the precedent (named {\it clusterizer V2}). Depending on the analysis type, one might want to use one or the other clusterization type. For this reason, a re-clusterization is also possible at the analysis level using as input the tower information stored in the analyzable data.

Once the cluster is defined, we calculate cluster parameters, shower shape parameters, that will help at the analysis level to identify each cluster as one particle type. Also, we compare the cluster position information with the propagation of tracks measured in the central barrel to the EMCAL surface, to identify the clusters generated by charged particles.

The final analysis objects, ESDs and AODs, contain all the cluster and cell basic informations allowing to redo the clusterization if needed at the analysis level.

The most expensive steps in the EMCal reconstruction are the pulse shape fitting  (this might not be the case when using fast fitters like HLT one) and the cluster-track matching (\bf{fshould we give numbers? this depends a lot on the pp, PbPb})

